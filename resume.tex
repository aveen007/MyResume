% resume.tex
% vim:set ft=tex spell:

\documentclass[9.5pt,letterpaper]{article}
\usepackage[letterpaper,margin=0.6in,top=.5in]{geometry}
\usepackage[utf8]{inputenc}
\usepackage{mdwlist}
\usepackage[T1]{fontenc}
\usepackage{textcomp}
\usepackage{tgpagella}
\usepackage{enumitem}
\usepackage{sectsty}
\newcommand{\resumeItemListEnd}{\end{itemize}\vspace{-12pt}}
\usepackage[hidelinks]{hyperref}
\pagestyle{empty}
\setlength{\tabcolsep}{0em}
% Adjust margins to fit more content
% \addtolength{\topmargin}{-0.6in}
\addtolength{\textheight}{0.3in}
% \addtolength{\oddsidemargin}{-0.5in}
% \addtolength{\evensidemargin}{-0.5in}
\addtolength{\textwidth}{0.2in}


\subsectionfont{\fontsize{10.5}{15}\selectfont}

% indentsection style, used for sections that aren't already in lists
% that need indentation to the level of all text in the document
\newenvironment{indentsection}[1]%
{\begin{list}{}%
		{\setlength{\leftmargin}{#1}}%
		\item[]%
	}
	{\end{list}}

% opposite of above; bump a section back toward the left margin
\newenvironment{unindentsection}[1]%
{\begin{list}{}%
		{\setlength{\leftmargin}{-0.6#1}}%
		\item[]%
	}
	{\end{list}}

% format two pieces of text, one left aligned and one right aligned
\newcommand{\headerrow}[2]
{\begin{tabular*}{\linewidth}{l@{\extracolsep{\fill}}r}
		#1 &
		#2 \\
\end{tabular*}}

% make "C++" look pretty when used in text by touching up the plus signs
\newcommand{\CPP}
{C\nolinebreak[4]\hspace{-.05em}\raisebox{.22ex}{\footnotesize\bf ++}}

% used for all new subsections
\newcommand{\resumesection}[1]
{\vspace{-2em}
\subsection*{\MakeUppercase{#1}}
\vspace{-0.2em}
\hrule
\vspace{0.2em}}

% lists used for most things
\newlist{experiencelist}{itemize}{1}
\setlist[experiencelist,1]{label=\textbullet,itemsep=-1pt,leftmargin=15pt,topsep=0pt,before=\vspace{-12pt}\fontsize{9.5}{11}\selectfont}


% and the actual content starts here
\begin{document}
	

\begin{indentsection}{0pt}
{\LARGE \textbf{Aveen Hussein}}
		
		

\ \ \href{mailto:aveen.hussein@niuitmo.ru}{\nolinkurl{aveen.hussein@niuitmo.ru}} 	 |Vyazemskiy Lane, 5-7 | Saint Petersburg, 197022| 
		7 (985) 794-9129|\ \ \href{https://github.com/aveen007}{\nolinkurl{github.com/aveen007}} 
\end{indentsection}
		

    
	\resumesection{Education}
	\begin{indentsection}{0pt}
		\hyphenpenalty=1000
		%\begin{description*}
        
		 \href{https://news.itmo.ru/en/education/students/news/13843/}{
    \headerrow
    {\textbf{ITMO University}}
    {Saint Petersburg, Russia}
}      \href{https://abit.itmo.ru/en/program/master/system_software}{
    \headerrow
    {\emph{Candidate for a Master's in System and Applied Software Engineering}}
    {June 2026 (expected)}
}
      
		 \begin{experiencelist} 
		 	\item \textbf{Thesis Topic:} Gamification of Urban Planning for Traffic Simulation and Prediction with Professor Ivan Perl.

            \item \textbf{Relevant Coursework:}  Multi-level organization of software, System Software, DevOps, Architecture of high-load applications.


    
		 \end{experiencelist}
            
		 \hyphenpenalty=1000
         
	\href{https://hiast.edu.sy/en/software_engineering_artificial_intelligence_program}{
    \headerrow
    {\textbf{Higher Institute of Applied Sciences and Technology}}
    {Damascus, Syria}
}
        \headerrow
		{\emph{B.Eng. in Software Engineering and Intelligent Systems }}
		{\emph{October 2019 -- June 2023}}
		 \begin{experiencelist} 
		 	\item \textbf{Relevant Coursework:} Machine Learning, Natural Language Processing, Computer Vision, Computer graphics.
% Multimedia , Object-Oriented Programming, Algorithms and Data Structures,, Parallel and Distributed Algorithms and Programs 
		 \end{experiencelist}
		 \hyphenpenalty=1000

	\end{indentsection}


%-----------EXPERIENCE-----------

	
	\resumesection{Academic Experience}
					\begin{indentsection}{0pt}
    \href{https://en.itmo.ru/en/faculty/121/Institute_of_Laser_Technologies.htm/}{
     \headerrow
             {\textbf{Institute of Laser Technology (ILT ITMO) }} 
             {Saint Petersburg, Russia}
         }
    \headerrow
    % Previously:
    {\emph{Computer Vision Engineer (Internship) }}
    {\emph{Oct. 2024 – Jan. 2025}}
		{ }
	\begin{experiencelist}
        
        \item Instance segmentation using Yolov8 training.
        \item Stable diffusion for synthetic image generation to use for dataset augmentation.
       
		\end{experiencelist}
\end{indentsection}

\vspace{-11 pt}
\begin{indentsection}{0pt}
 \headerrow{
 
    \emph{Graduate Research Assistant }}
        {Jan. 2025 – Present}
        
        Developed and integrated a monitoring and control system for automation of production processes using the
        example of laser welding, the system consisted of:       
        \vspace{1.1em}

        
	\begin{experiencelist}
    {
        \item {Instance segmentation using Yolov8 on a custom data set to automatize identifying weld defects.}
        \item{Using optical character recognition (OCR) and OpenCV edge detection algorithms to collect data from weld images.}}
       
	\end{experiencelist}
    \end{indentsection}
		 	\begin{indentsection}{0pt}           
     \href{https://cosm-lab.science/ru/}{
		\headerrow
            {
            \textbf{COSM Lab}}
		{Saint Petersburg, Russia}
            }
    \headerrow
    {\emph{Graduate Research Assistant}}
    {Sept. 2024 – Present}
		\begin{experiencelist}
			\item Developing a large-scale online city-building simulator.

		\end{experiencelist}
	% 				% \begin{indentsection}{0pt}
	% \href{http://www.ncd.sy/}{
	% \headerrow
 %    {\textbf{Teacher}}
 %    {Latakia, Syria}
 %    }
	% 	\headerrow
	% 	{\emph{National Center for the Distinguished}}
	% 	{\emph{Fall 2017}}
	% 	\begin{experiencelist}
 %        \item Physics teacher for the Syrian team participating in the International Physics Olympiad IPHO
 %       	\end{experiencelist}
\end{indentsection}



	    
%-----------EXPERIENCE-----------


	
	

			\resumesection{Selected Projects}
\begin{indentsection}{0pt}

            
        \href{https://drive.google.com/file/d/1cUS0d8aa4VFpjxiumKsFZCbVPzJMD2SW/view?usp=sharing}{
    \headerrow{
    \parbox{0.8\linewidth}
		{\textbf{Intelligent Weld Inspection: Machine Learning-Driven Defect
Detection for Automated Laser Welding Quality Control}}
    }
		% {Damascus, Syria}
        {\emph{Oct-2024, present}}
		% \vspace{3em}        
        }\headerrow
% \vspace{-10 pt}



% Add this to the "Experience" or "Projects" section of your LaTeX resume
\begin{experiencelist}
    \item Developed an automated weld defect detection system using \textbf{YOLOv8 instance segmentation}, achieving \textbf{97.5\% accuracy} in defect classification and geometric measurements.
    
    \item Addressed data scarcity by fine-tuning \textbf{Stable Diffusion models} to generate \textbf{4,537 synthetic weld images}, improving model generalization for rare defects. Published findings in:
    \begin{itemize}
        \item \textit{Augmentation of Laser Welding Dataset through a Combination of Evolutionary Optimization and Deep Learning}. GECCO 2025.
    \end{itemize}
    
    \item Designed a \textbf{computer vision pipeline} for weld geometry analysis, extracting key measurements (width, depth, misalignment) with \textbf{sub-millimeter precision}.
    
    % \item Reduced manual inspection time from \textbf{minutes to microseconds}, enhancing efficiency in industrial quality control.
    
    % \item Collaborated with welding experts to refine model performance over \textbf{7 training iterations}, resolving edge detection and false-negative challenges.
    
    \item Published full pipeline research in:
    \begin{itemize}
        \item \textit{Intelligent Weld Inspection: Machine Learning-Driven Defect Detection for Automated Laser Welding Quality Control}. FLAMMN 2025 (accepted).
    \end{itemize}
\end{experiencelist}
       \end{indentsection}

\begin{indentsection}{0pt}
           \href{ https://github.com/aveen007/autonomous-vehicle-with-unity}{
               \headerrow
		{\textbf{Self-driving car in a simulated environment using deep-reinforcement learning}}
        {\emph{March 2022,Sept 2023} }
		% {Damascus, Syria}
        }
               \headerrow
        
		
        \vspace{0em}  
        \begin{indentsection}{0pt}
		\begin{experiencelist}
		    
		
        \item{
         Led multidisciplinary research and coordinated the work with 6 supervisors from diverse backgrounds in Computer
Vision, Graphics, AI, and Systems Engineering. 
        }
        \item{
         Achieved a 94\% evaluation rating for the project, which was
recognized as the top graphics initiative within the institution for a bachelor's thesis.
        }
        \item{Designed and developed a custom physics system by simulating 6 forces that affect the car. Simulated the car with 202 Lidar sensors and 2 Cameras using Unity's ML-Agents.}
\item{Implemented a dynamic traffic system using a way-point algorithm and a pool system and applied it to add up to 30 cars and 30 passengers.}
 
\item{Compared algorithms DQN and DDQN for car control and proposed a reward function and tested it with success in achieving 30 km/h average car speed after training using DQN.} 
\item{Implemented a first-ever integration of cutting-edge computer-vision algorithms CLR-NET and YOLO-v8 into unity to allow for lane and object detection.}
\item{Integrated Open-CV library into unity by creating a custom side channel to connect the training and the simulation.}
\item{Designed and implemented user interfaces to easily track the lane detection and object detection process in real time.}
		\end{experiencelist}
  
\end{indentsection}
 
        

% \begin{indentsection}{0pt}
        
%         \href{https://www.youtube.com/watch?v=BP_n_MwNSS4&t}{
%     \headerrow
% 		{\textbf{Take Aim (A Multiplayer First-Person Shooting Game with Unity Engine)}}
% 		% {Damascus, Syria}
%         {\emph{July-2022, Sep-2022}}
% 		\vspace{-1em}        
%         }\headerrow
% % \vspace{-10 pt}



%     \emph Designed and developed a multiplayer game supporting up to 30 players using the Photon service.
%         \begin{itemize}
%     \item Created and implemented two terrains.
%     \item Designed visual and sound effects for the game.
%     \item Developed user interface and user experience (UI/UX) elements.
%     \item Designed and implemented eight different weapons using the Unity Engine.
%     \item Completed the project as a 4\textsuperscript{th}-year individual project at HIAST and received a 92\% evaluation from the judging committee.
% \end{itemize}

 
        
%         % \emph Designed and developed the game with a client-server multiplayer quality of up to 30 players using Photon service. Designed and implemented 2 terrains, visual and sound effects, UI/UX, 8 different weapons using Unity Engine, and received 92\% evaluation for my work by the judging committee as a 4th year individual project in HIAST.
% \end{indentsection}




				
	
	\resumesection{Skills}
	\begin{indentsection}{0pt}
		\fontsize{9.5}{11}\selectfont
		
		\hyphenpenalty=1000
			
		{\textbf{Programming Languages.} C\#, \CPP , C, Assembly, Python, Java, JavaScript, PHP}
		
		\vspace{-0.2em}
		
		\textbf{Technologies.} PyTorch, Docker, Unity, Unreal Engine, Pandas,  Dask, Statsmodels, scikit-learn,  Flask
		
			\textbf{Languages.} Arabic (Native), English (C2), Russian (B2), French (B1)
		
		\vspace{1em}
		
		
                      \end{indentsection}


		
       \end{indentsection}
	
\end{document}